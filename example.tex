\documentclass{umlthesis}
\usepackage{lipsum}
\usepackage{newtxtext}
\usepackage{newtxmath}

\title{A Fascinating Thesis}
\author{Stew Dent}
\department{Physics and Applied Physics}
\degree{Doctor of Philosophy}
%% \concentration{}                % Default: Same as \department below
\previousdegree{B.S.}{Alma Mater}{2012}
\previousdegree{M.S.}{Some Other Place}{2016}
\supervisor{Super Visor}
%% \supervisordegree{}             % Default: Ph.D.
%% \supervisordepartment{}         % Default: Same as \department
%% \supervisortitle{}              % Default: Thesis Supervisor
%% \chair{}  % This doesn't seem to actually be necessary, looking at examples in the Thesis Guide....
\reader{Quick Reader}
\reader{Slow Reader}






\begin{document}
\maketitle
%%%%%%%%%%%%%%%%%%%%%%%%%%%%%%%%%%%%%%%%
% \begin{abstract}
%   This example document will utilize some of the features of \texttt{umlthesis.cls}.
% \end{abstract}

%%%%%%%%%%%%%%%%%%%%%%%%%%%%%%%%%%%%%%%%
\begin{acknowledgments}
An acknowledgments page is optional.
\end{acknowledgments}

%%%%%%%%%%%%%%%%%%%%%%%%%%%%%%%%%%%%%%%%
\tableofcontents
\listoffigures
\listoftables

%%%%%%%%%%%%%%%%%%%%%%%%%%%%%%%%%%%%%%%%
%%%%%%%%%%%%%%%%%%%%%%%%%%%%%%%%%%%%%%%%
\chapter{Examples}
This chapter will show as much of the document class\footnote{This is a short footnote.} as possible. The footnotes\footnote{You'll find that this footnote is overly verbose and won't fit on a single line. If I have written the document class correctly, then the indentation will match the specification in the UML Thesis Guide.} are required to be indented in a particular way, for example.

\begin{quote}
  This is a paragraph quoted from somewhere else. It must be single-spaced and indented on both sides.
\end{quote}

In Table~\ref{tab:fruits} you can see what a table will look like. Refer to Figure~\ref{fig:square} to see a figure.

\begin{table}[h]
  \centering
  \caption[Comparison of fruits]{Only fruits of the same type may be compared safely.}
  \label{tab:fruits}
  \begin{tabular}{l|cc}
    & Apples & Oranges \\
    \hline
    Apples & yes & no \\
    Oranges & no & yes \\
  \end{tabular}
\end{table}

\lipsum[1]

\begin{figure}
  \centering
  \rule{2in}{2in}
  \caption{A black square.}
  \label{fig:square}
\end{figure}

\section{A Section}
This is what a main section heading looks like.

\lipsum[1]

\subsection{A Subsection}
Sub sections look like this.

\lipsum[1]

\subsubsection{A Sub-subsection (Don't go this deep!)}
Don't use sub-subsections.

\lipsum[1]

\section{Another Section}
This proves that the section numbering works.

%%%%%%%%%%%%%%%%%%%%%%%%%%%%%%%%%%%%%%%%
\chapter{Requirements}

\newcommand{\pkg}[1]{\textsf{#1}}

This document class requires \pkg{natbib},  \pkg{setspace}, and \pkg{tocloft}, which are probably already a part of your \LaTeX\ distribution.

%%%%%%%%%%%%%%%%%%%%%%%%%%%%%%%%%%%%%%%%
\nocite{*}
\bibliographystyle{plainnat}
\bibliography{example}

%%%%%%%%%%%%%%%%%%%%%%%%%%%%%%%%%%%%%%%%
\appendix
\chapter{Appendix Chapter}
\lipsum[2]

\end{document}
